\begin{python}
# Only execute if the _matlab singleton exists:
if not(_matlab == None):
    columns = [
        {'key': 'SECONDS'},
        {'key': 'T_01'},
        {'key': 'T_02'}
    ]
    _matlab.load_dataset(f'{_data}/example.csv', _name, columns)
    _matlab.eval_template(
        f'{_templates}/matlab/figure.m',
        height=300
    )
    _matlab.eval_template(
        './a_matlab_plot.m',
        name=_name
    )
    _matlab.eval_template(
        f'{_templates}/matlab/axis.m',
        xlabel='Time [s]',
        ylabel='Temperature [\u00b0C]',
        xmin=0,
        xmax=10,
        xtickminor=1,
        xtick=5,
        ymin=30,
        ymax=40,
        ytickminor=1,
        ytick=5
    )
    _matlab.eval_template(
        f'{_templates}/matlab/pdf.m',
        pdfpath=f'{_build}/images/{_name}.pdf'
    )
figure = _templater.render(
    f'{_templates}/latex/figure.tex',
    label=f'fig:{_name}',
    caption='A MATLAB Plot',
    path=f'images/{_name}'
)
print(figure)

\end{python}
The following shows the accompanying MATLAB template "a\_matlab\_plot.m". Note that the variable "name" within the double curly brackets is replaced with the parameter set in eval\_template in the .py script above (for information on Jinja templates see Section \ref{sec:jinja}).

\begin{matlab}
hold on;
p1 = plot( ...
    {{name}}.SECONDS, ...
    {{name}}.T_01, ...
    'Color', 'k',  ...
    'LineWidth', 1 ...
);
p2 = plot( ...
    {{name}}.SECONDS, ...
    {{name}}.T_02, ...
    '--', ...
    'Color', 'k', ...
    'LineWidth', 1 ...
);
title('A MATLAB Plot');
legend([p1, p2], 'Temperature 1', 'Temperature 2');
hold off

\end{matlab}

